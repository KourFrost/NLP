%-*- Mode:LaTeX; -*-      
\documentclass[11pt]{article}
\usepackage{vmargin}		% Force narrower margins
\setpapersize{USletter}
\setmarginsrb{1.0in}{1.0in}{1.0in}{0.6in}{0pt}{0pt}{0pt}{0.4in}
\setlength{\parskip}{.1in}  % removed space between paragraphs
\setlength{\parindent}{0in}

\usepackage{epsfig}
\usepackage{graphicx}
\usepackage{xcolor}
\newcommand{\ra}{$\rightarrow$~}
\newcommand{\dt}{$\circ$~}

\begin{document}

\large
\begin{center}
{\bf CS-5340/6340, Written Assignment \#1} \\
{\bf DUE: Tuesday, September 7, 2021 by 11:59pm} \\
{\bf  \textcolor{red}{by Jacob Herrmann u0259542}}\\ ~ \\
{\bf  Submit your assignment on CANVAS in pdf format.}
\end{center}
\normalsize

\begin{enumerate}  


\item (30 pts) For each sentence below, label each word with its
  correct part-of-speech (POS) tag based upon the word's use in the sentence.
  Do not assign POS tags to punctuation marks. 

  Choose from the following list of part-of-speech tags: {\bf
    adjective ({\sc adj}), adverb ({\sc adv}), article ({\sc art}),
    conjunction ({\sc conj}), gerund ({\sc ger}), infinitive ``to''
    ({\sc inf}), modal verb ({\sc mod}), noun ({\sc noun}), particle
    ({\sc part}), preposition ({\sc prep}), 
    pronoun ({\sc pro}), 
    verb ({\sc verb})} [not modal]. 

For infinitive verb phrase constructions, label ``to'' as {\sc inf}
and the verb itself as {\sc verb}. \par

Please show your part-of-speech tag assignments by appending a
slash and POS tag after each word. For example: ``Natural/{\sc adj}
language/{\sc noun} is/{\sc verb} fun/{\sc adj}.'' 

\begin{enumerate}

\item An engineer blew up a huge tower with dynamite.
\par An\textcolor{red}{{/\sc art}} engineer\textcolor{red}{{/\sc noun}} blew\textcolor{red}{{/\sc verb}} up\textcolor{red}{{/\sc part}} a\textcolor{red}{{/\sc art}} huge\textcolor{red}{{/\sc adv}} tower\textcolor{red}{{/\sc noun}} with\textcolor{red}{{/\sc prep}} dynamite\textcolor{red}{{/\sc noun}}.

\item Snowbird will be hosting skiing events tomorrow.
\par Snowbird\textcolor{red}{{/\sc noun}} will\textcolor{red}{{/\sc mod}} be\textcolor{red}{{/\sc verb}} hosting\textcolor{red}{{/\sc ger}} skiing\textcolor{red}{{/\sc ger}} events\textcolor{red}{{/\sc noun}}\\* tomorrow\textcolor{red}{{/\sc noun}}.

\item Susan reluctantly asked to borrow money. 
\par Susan\textcolor{red}{{/\sc noun}} reluctantly\textcolor{red}{{/\sc adv}} asked\textcolor{red}{{/\sc verb}} to\textcolor{red}{{/\sc inf}} borrow\textcolor{red}{{/\sc verb}} money\textcolor{red}{{/\sc noun}}. 

\item Swimming with sharks is incredibly dangerous.
\par Swimming\textcolor{red}{{/\sc ger}} with\textcolor{red}{{/\sc prep}} sharks\textcolor{red}{{/\sc noun}} is\textcolor{red}{{/\sc verb}} incredibly\textcolor{red}{{/\sc adv}} dangerous\textcolor{red}{{/\sc adj}}.

\item George would not talk to people at the bowling alley. 
\par George\textcolor{red}{{/\sc noun}} would\textcolor{red}{{/\sc mod}} not\textcolor{red}{{/\sc adv}} talk\textcolor{red}{{/\sc verb}} to\textcolor{red}{{/\sc part}} people\textcolor{red}{{/\sc noun}} at\textcolor{red}{{/\sc prep}} the\textcolor{red}{{/\sc art}} bowling\textcolor{red}{{/\sc noun}} alley\textcolor{red}{{/\sc noun}}. 

\item She agreed to buy it but must pay cash. 
\par She\textcolor{red}{{/\sc pro}} agreed\textcolor{red}{{/\sc verb}} to\textcolor{red}{{/\sc inf}} buy\textcolor{red}{{/\sc verb}} it\textcolor{red}{{/\sc pro}} but\textcolor{red}{{/\sc conj}} must\textcolor{red}{{/\sc mod}} pay\textcolor{red}{{/\sc verb}} \\*cash\textcolor{red}{{/\sc noun}}. 

\item Jane called off the creative writing project. 
\par Jane\textcolor{red}{{/\sc noun}} called\textcolor{red}{{/\sc verb}} off\textcolor{red}{{/\sc part}} the\textcolor{red}{{/\sc art}} creative\textcolor{red}{{/\sc adj}} writing\textcolor{red}{{/\sc ger}} project\textcolor{red}{{/\sc noun}}. 

\item Tom dressed up in a tuxedo. 
\par Tom\textcolor{red}{{/\sc noun}} dressed\textcolor{red}{{/\sc verb}} up\textcolor{red}{{/\sc part}} in\textcolor{red}{{/\sc prep}} a\textcolor{red}{{/\sc art}} tuxedo\textcolor{red}{{/\sc noun}}. 

\end{enumerate}


\newpage
\item (20 pts) For each sentence below, indicate whether the verb
  phrase is in an {\bf active voice} or {\bf passive voice}
  construction.  

\begin{enumerate}

\item The dog has been chasing a squirrel around the yard for hours. \textcolor{red}{{\bf active voice}}

\item The lawn has not been mowed in a year. \textcolor{red}{{\bf active voice}}

\item The man discovered a grizzly bear by the lake. \textcolor{red}{{\bf active voice}}

\item The vegetables were cooked on the outdoor grill. \textcolor{red}{{\bf passive voice}}

\item The war was fought over 50 years ago. \textcolor{red}{{\bf passive voice}}

\item He had forgotten about the concert. \textcolor{red}{{\bf active voice}}

\item She felt great pride in her son's award-winning artwork. \textcolor{red}{{\bf active voice}}

\item Kathy plans to be an architect. \textcolor{red}{{\bf active voice}}

\item Jim's electricity bill will be paid in full by his mom. \textcolor{red}{{\bf passive voice}}

\item The missing artwork might have been stolen. \textcolor{red}{{\bf passive voice}}

\end{enumerate}



\newpage
\item (30 pts) For each sentence below: 

\begin{enumerate}
\item[(1)] Identify the noun phrases (NPs) that correspond to the syntactic roles
  of {\bf Subject (SUBJ)}, {\bf Direct Object (DOBJ)}, and {\bf
    Indirect Object (IOBJ)}
  with respect to the verb phrase.   Put brackets [] around each
  relevant NP followed by a slash (/) and the syntactic role. For
  example: {\it [Natural Language]/SUBJ is fun. }
  Note that each sentence will have at least
  one of the syntactic roles, but not necessarily all of them!

\item[(2)]
 Indicate whether the main verb appears in an {\bf intransitive} construction, a {\bf transitive} construction, or a
  {\bf ditransitive} construction. Only give the answer {\bf
    transitive} if the usage is \underline{not} {\bf ditransitive}. \\
\end{enumerate}


\begin{enumerate}

\item George wrote his wife a poem. 
\par \textcolor{red}{[George]/SUBJ} wrote \textcolor{red}{[his wife]/IOBJ} \textcolor{red}{[a poem]/DOBJ}.
\\* \textcolor{red}{{\bf ditransitive}}
\item The boy tossed a red ball to his dog. 
\par \textcolor{red}{[The boy]/SUBJ} tossed \textcolor{red}{[a red ball]/IOBJ} to \textcolor{red}{[his dog]/DOBJ}.
\\* \textcolor{red}{{\bf ditransitive}}
\item John received several urgent emails from his boss. 
\par \textcolor{red}{[John]/SUBJ} received \textcolor{red}{[several urgent emails]/IOBJ} from \textcolor{red}{[his boss]/DOBJ}his boss.
\\* \textcolor{red}{{\bf ditransitive}}
\item Sarah ordered a pizza with pepperoni. 
\par \textcolor{red}{[Sarah]/SUBJ} ordered \textcolor{red}{[a pizza with pepperoni]/DOBJ}. 
\\* \textcolor{red}{{\bf transitive}}
\item The man sneezed into his hankerchief.
\par \textcolor{red}{[The man]/SUBJ} sneezed into \textcolor{red}{[his hankerchief]/DOBJ} his hankerchief.
\\* \textcolor{red}{{\bf transitive}}
\item Dark smoke was seen over the wilderness area.
\par \textcolor{red}{[Dark smoke]/SUBJ} was seen over \textcolor{red}{[the wilderness area]/DOBJ}.
\\* \textcolor{red}{{\bf transitive}}
\item Fran told the kids a bedtime story.
\par \textcolor{red}{[Fran]/SUBJ} told \textcolor{red}{[the kids]/IOBJ} \textcolor{red}{[a bedtime story]/DOBJ}. 
\\* \textcolor{red}{{\bf ditransitive}}
\item She transferred money to her sister.
\par She transferred \textcolor{red}{[money]/IOBJ} to \textcolor{red}{[her sister]/DOBJ}.
\\* \textcolor{red}{{\bf ditransitive}}
\item Pedro sat quietly at his desk. 
\par \textcolor{red}{[Pedro]/SUBJ} sat quietly at \textcolor{red}{[his desk]/DOBJ}. 
\\* \textcolor{red}{{\bf transitive}}
\item The company president guaranteed Mary a job. 
\par \textcolor{red}{[The company president]/SUBJ} guaranteed \textcolor{red}{[Mary]/IOBJ} \textcolor{red}{[a job]/DOBJ}. 
\\* \textcolor{red}{{\bf ditransitive}}
\end{enumerate}


\newpage
\item (20 pts) Consider the document collection below, which consists of
5 (short!) documents: \\ 

\noindent
D1: {\it squirrels eat nuts} \\
D2: {\it the dog chased a squirrel up a tree} \\
D3: {\it the squirrel ate a nut} \\
D4: {\it a raccoon was chasing the squirrel} \\
D5: {\it the dog often chases raccoons up trees} \\

Create an inverted file structure for this document collection
using the true morphological root (lemma) of each word as the indexed
term. Include all words in the sentences except you do not need to
index the words ``a'' and ``the''. You do \underline{not} need to
include location (proximity) information. 

\begin {enumerate}
\color{red}
\item squirrel: (D1, D2, D3, D4)
\item eat: (D1)
\item nut: (D1, D3)
\item dog: (D2, D5)
\item chas: (D2, D4, D5 )
\item up: (D2, D5)
\item tre: (D2, D5)
\item at: (D3)
\item raccoon (D4, D5)
\item wa (D4)
\item often (D5)
\end{enumerate}

\newpage
\underline{\textbf{Question \#5 is for CS-6340 students ONLY!}}  \\

\item (12 pts) Fill in the table below with morphology rules to derive all
  of the words below from the specified root form {\it in a
    linguistically sensible way}. Some derivations may require the
  application of multiple rules. In this case, put each 
  rule in a separate row of the table. Also, some words may have
  multiple derivations. Be sure to include \underline{all} derivations that make
  sense.

  For illustration, the table is already filled in with the derivation
  of {\it ``unfairly''} from the root {\it ``fair''}. 

\begin{enumerate}

\item speculative (root = ``speculate'')

\item indigestible (root = ``digest'')

\item understandability (root = ``understand'')

\item hyperemotional (root = ``emotion'')

\item counterclockwise (root = ``clock'')

\item intensification (root = ``intense'')

\end{enumerate}


  \begin{center}
 \small
  \begin{tabular}{|l|l||l|l|l|l||l|} \hline
  \textbf{Derived} & \textbf{Origin} & \textbf{Prefix} & \textbf{Suffix} & 
  \textbf{Replace} & \textbf{POS of} & \textbf{POS of} \\
  \textbf{Word} & ~ & ~ & ~ & \textbf{Chars} & \textbf{Origin} &
  \textbf{Derived} \\ \hline
   unfairly & unfair & - & ly  & - & ADJ & ADV \\
   unfair & fair & un & - & - & ADJ & ADJ \\  \hline \hline
   \textcolor{red}{speculative}& \textcolor{red}{speculate}& \textcolor{red}{-} & \textcolor{red}{ive}  & \textcolor{red}{e}& \textcolor{red}{ADJ} & \textcolor{red}{VERB}\\  \hline \hline
   \textcolor{red}{indigestible} & \textcolor{red}{digestible} & \textcolor{red}{in} & \textcolor{red}{-} & \textcolor{red}{-} & \textcolor{red}{ADJ} & \textcolor{red}{ADJ} \\
   \textcolor{red}{digestible} & \textcolor{red}{digest} & \textcolor{red}{-} & \textcolor{red}{ible} & \textcolor{red}{-} & \textcolor{red}{ADJ} & \textcolor{red}{VERB} \\
   \textcolor{red}{indigestible} & \textcolor{red}{indigest} & \textcolor{red}{-} & \textcolor{red}{ible} & \textcolor{red}{-} & \textcolor{red}{ADJ} &  \textcolor{red}{VERB} \\
   \textcolor{red}{indigest} & \textcolor{red}{digest} & \textcolor{red}{in} & \textcolor{red}{-} & \textcolor{red}{-} & \textcolor{red}{VERB} & \textcolor{red}{VERB} \\ \hline \hline
   \textcolor{red}{understandability}& \textcolor{red}{understand} & \textcolor{red}{-} & \textcolor{red}{ability} & \textcolor{red}{-} & \textcolor{red}{NOUN} & \textcolor{red}{VERB} \\ \hline \hline
   \textcolor{red}{hyperemotional} & \textcolor{red}{emotional} & \textcolor{red}{hyper}& \textcolor{red}{-} & \textcolor{red}{-} & \textcolor{red}{ADJ} & \textcolor{red}{ADJ} \\
   \textcolor{red}{emotional} & \textcolor{red}{emotion} & \textcolor{red}{-}& \textcolor{red}{al} & \textcolor{red}{-} & \textcolor{red}{ADJ} & \textcolor{red}{NOUN} \\
   \textcolor{red}{hyperemotional} & \textcolor{red}{hyperemotion} & \textcolor{red}{-}& \textcolor{red}{al} & \textcolor{red}{-} & \textcolor{red}{ADJ} & \textcolor{red}{NOUN} \\
   \textcolor{red}{hyperemotion} & \textcolor{red}{emotion} & \textcolor{red}{hyper}& \textcolor{red}{-} & \textcolor{red}{-} & \textcolor{red}{NOUN} & \textcolor{red}{NOUN} \\ \hline \hline
   \textcolor{red}{counterclockwise} & \textcolor{red}{clockwise} & \textcolor{red}{counter} & \textcolor{red}{-}  & \textcolor{red}{-}  &\textcolor{red}{ADV} & \textcolor{red}{ADV}  \\
   \textcolor{red}{clockwise} & \textcolor{red}{clock} & \textcolor{red}{-} & \textcolor{red}{wise}  & \textcolor{red}{-}  &\textcolor{red}{ADV} & \textcolor{red}{ADV}  \\ \hline \hline
   \textcolor{red}{intensification}& \textcolor{red}{intense}& \textcolor{red}{-} & \textcolor{red}{ification}  & \textcolor{red}{e}& \textcolor{red}{NOUN} & \textcolor{red}{ADJ}\\  \hline \hline

   ~ & ~ & ~ & ~ & ~ & ~ & ~ \\
 ~ & ~ & ~ & ~ & ~ & ~ & ~ \\
 ~ & ~ & ~ & ~ & ~ & ~ & ~ \\
 ~ & ~ & ~ & ~ & ~ & ~ & ~ \\
 ~ & ~ & ~ & ~ & ~ & ~ & ~ \\
 ~ & ~ & ~ & ~ & ~ & ~ & ~ \\
 ~ & ~ & ~ & ~ & ~ & ~ & ~ \\
 ~ & ~ & ~ & ~ & ~ & ~ & ~ \\
 ~ & ~ & ~ & ~ & ~ & ~ & ~ \\
 ~ & ~ & ~ & ~ & ~ & ~ & ~ \\
 ~ & ~ & ~ & ~ & ~ & ~ & ~ \\
 ~ & ~ & ~ & ~ & ~ & ~ & ~ \\
 ~ & ~ & ~ & ~ & ~ & ~ & ~ \\
 ~ & ~ & ~ & ~ & ~ & ~ & ~ \\
 ~ & ~ & ~ & ~ & ~ & ~ & ~ \\ \hline
  \end{tabular}
  \end{center}


\end{enumerate}
% END OF WRITTEN QUESTIONS

\end{document}

