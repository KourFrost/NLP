%-*- Mode:LaTeX; -*-      
\documentclass[11pt]{article}
\usepackage{vmargin}		% Force narrower margins
\usepackage{multirow}
\setpapersize{USletter}
\setmarginsrb{1.0in}{1.0in}{1.0in}{0.6in}{0pt}{0pt}{0pt}{0.4in}
\setlength{\parskip}{.1in}  % removed space between paragraphs
\setlength{\parindent}{0in}
\usepackage{xcolor}
\usepackage{epsfig}
\usepackage{graphicx}
\newcommand{\ra}{$\rightarrow$~}
\newcommand{\dt}{$\circ$~}

\begin{document}

\large
\begin{center}
{\bf CS-5340/6340, Written Assignment \#3} \\
{\bf DUE: Tuesday, November 23, 2021 by 11:59pm} \\ ~ \\
{\bf  Submit your assignment on CANVAS in pdf format.}
\end{center}
\normalsize

\begin{enumerate}  


\item (20 pts) Show the instantiated patterns that would be
  generated by the AutoSlog pattern generator when given each sentence
  below as input, where the {\bf bold-faced} noun phrase (NP) has been
  labeled with the event role indicated after the slash. 

For example, ``{\bf (John Smith)/VICTIM} was shot'' indicates that
``John Smith'' was labeled as a VICTIM by a human.  \\

\begin{enumerate}

\item {\bf (The tourist)/VICTIM} was robbed in Spain by a young man. \\
\textcolor{red}{$<$VICTIM$>$ was robbed}


\item The tourist was robbed in Spain by {\bf (a young
    man)/PERPETRATOR}. \\
\textcolor{red}{by $<$PERPETRATOR$>$}


\item {\bf (XYZ Co.)/SELLER} tried to sell ABC Inc. for 20 million
  dollars. \\
\textcolor{red}{$<$SELLER$>$ to sell}

\item XYZ Co. tried to sell {\bf (ABC Inc.)/SELLEE} for 20 million dollars. \\
\textcolor{red}{to sell$<$SELLEE$>$}

\item XYZ Co. tried to sell ABC Inc. for {\bf (20 million
    dollars)/PRICE}. \\
\textcolor{red}{for $<$PRICE$>$}

\item {\bf (A tall woman)/PERPETRATOR} with a bomb was seen near the enormous
\\  explosion of a clothing factory. \\
\textcolor{red}{$<$PERPETRATOR$>$ with a bomb}

\item A tall woman with {\bf (a bomb)/WEAPON} was seen near the
  enormous explosion of a clothing  factory. \\
\textcolor{red}{with $<$WEAPON$>$}

\item A tall woman with a bomb was seen near the enormous explosion of {\bf (a
    clothing \\ factory)/TARGET}. \\
\textcolor{red}{explosion of $<$TARGET$>$}

\end{enumerate}



\newpage

\item (35 pts) For each sentence below, label each noun phrase (NP)
  with the appropriate thematic role. Put parentheses around the NP
  and indicate the thematic role with a slash, for example: (Mickey
  Mouse)/AGENT.

\begin{enumerate}

\item George flew to Las Vegas by helicopter.\\
\textcolor{red}{(George)/AGENT flew to (Las Vegas)/TO-LOC by (helicopter)/INSTRUMENT}
\item Jim built his children a treehouse with his brother. \\
\textcolor{red}{(Jim)/AGENT built (his children)/RECIPIENT (a treehouse)THEME with (his brother)/CO-AGENT.}
\item The young girl raced with her sister around the yard. \\
\textcolor{red}{(The young girl)/AGENT raced with (her sister)/CO-AGENT around (the yard)/THEME.}

\item The waiter served Charlie a salad with Italian dressing.\\
\textcolor{red}{(The waiter)/AGENT served (Charlie)/RECIPIENT (a salad)/THEME with (Italian dressing)/INSTRUMENT.}

\item The ballerina taught us dancing.\\
\textcolor{red}{(The ballerina)/AGENT taught (us)/BENEFICIARY (dancing)/THEME.}

\item Two security cameras recorded the burglary. \\
\textcolor{red}{(Two security cameras)/AGENT recorded (the burglary)/THEME.}

\item Three collared cougars were tracked by officials with GPS devices.\\
\textcolor{red}{(Three collared cougars)/THEME were tracked by (officials)/AGENT with (GPS devices)/INSTRUMENT.}

\item Susan fixed the broken cabinet for her ailing grandmother. \\
\textcolor{red}{(Susan)/AGENT fixed (the broken cabinet)/THEME for (her ailing grandmother)/RECIPIENT.}

\item Ted bought a kitchen table with chairs with his wife. \\
\textcolor{red}{(Ted)/AGENT bought (a kitchen table)/THEME with (chairs)/CO-THEME with (his wife)/CO-AGENT. }

\item Jim's cancer was cured by chemotherapy.\\
\textcolor{red}{(Jim's cancer)/THEME was cured by (chemotherapy)/AGENT.}

\item Cathy dislikes country music. \\
\textcolor{red}{(Cathy)/AGENT dislikes (country music)/THEME.}

\item Skiing competitions will be held in northern Utah. \\
\textcolor{red}{(Skiing competitions)/AGENT will be held in (northern Utah)/AT-LOC.}\\

\end{enumerate}

\newpage

\item (12 pts) Consider the (tiny!) text corpus below, which contains
  5 documents (D1-D5). Each document consists of 2 sentences, so the
  entire corpus contains 10 sentences (S1-S10). For all computations
  below, treat the words as case-insensitive (e.g., ``Fish'' is the
  same as ``fish'').  Use log base 2 for the computations. {\bf Show
    all your work!}

\begin{center}
\begin{tabular}{|l|l|} \hline
\multirow{2}*{D1} & S1: Nearly all birds can fly. \\
& S2: Penguins are water birds but can not fly. \\ \hline
\multirow{2}*{D2} & S3: Penguins eat fish and swim in the ocean. \\
& S4: Pelicans are birds that also eat fish and fly. \\ \hline
\multirow{2}*{D3} & S5: Pelicans and penguins eat ocean fish. \\
& S6: Fish swim in many rivers. \\ \hline
\multirow{2}*{D4} & S7: Fish can swim in both the ocean and rivers. \\
& S8: Most water birds eat fish in the ocean or rivers.  \\  \hline
\multirow{2}*{D5} & S9: Some birds can swim, such as King penguins and
Emperor penguins. \\
& S10: Penguins swim far into the ocean to catch deep sea squid.\\ \hline
\end{tabular}
\end{center}

\vspace*{.1in}
\begin{description}
\item[(a)] Compute TF-IDF(penguins, D5)\\
\textcolor{red}{
$TF-IDF = TF(penguins) *IDF(penguins, D5)$ \\
$TF(penguins,D5)= 3$\\
$IDF(penguins)= log_2 {\frac{5}{4}}= 0.3219$\\
$TF-IDF = 3*0.3219 = 0.9657$ \\}

\item[(b)] Compute TF-IDF(fish, D2)
\textcolor{red}{ 
$TF-IDF = TF(fish) *IDF(fish, D2)$ \\
$TF(fish,D2)=2$\\
$IDF(fish)= log_2{\frac{5}{3}}= 0.7369$\\
$TF-IDF = 2*0.7369 = 1.4738$ \\ }
\end{description}

Use the following definitions for the PMI computations below. Let x and y
be terms in the vocabulary. Define P(x) as the probability that a
sentence will contain at least one instance of x. Define P(x,y) as the
probability that a sentence will contain at least one instance of x
and at least one instance of y. 

\begin{description}
\item[(c)] Compute PMI(penguins,fish)\\
\textcolor{red}{ 
$PMI(penguins,fish) = log_2{\frac{P(penguins,fish)}{P(penguins)*P(fish)}}$\\
$P(penguins,fish) = \frac{2}{10}= 0.2$\\
$P(penguins)= \frac{5}{10}= 0.5$\\
$P(fish)= \frac{6}{10}= 0.6$\\
$PMI(penguins,fish) = log_2{\frac{0.2}{0.5*0.6}}= -0.5849$ \\ }

\item[(d)] Compute PMI(birds, fly)\\
\textcolor{red}{ 
$PMI(birds,fly) = log_2{\frac{P(birds,fly)}{P(birds)*P(fly)}}$\\
$P(birds,fly) = \frac{3}{10}= 0.3$\\
$P(birds)= \frac{5}{10}= 0.5$\\
$P(fly)= \frac{3}{10}= 0.3$\\
$PMI(birds,fly) = log_2{\frac{0.3}{0.5*0.3}}= 1$ \\ }
\end{description}
 

\newpage

\item (8 pts) Consider the following two  vectors:
\begin{quote}
{\bf X} = $<$7, 4, 11, 2, 9$>$ \\
{\bf Y} = $<$1, 8, 10, 3, 6$>$ 
\end{quote}

\vspace*{.2in}
Show all of your work for the computations below. 

\begin{enumerate}
\item Compute the similarity between {\bf X} and {\bf Y} using
  Manhattan Distance. \\
  \textcolor{red}{$ |7-1|+|4-8|+|11-10|+|2-3|+|9-6|=15$}


\item Compute the similarity between {\bf X} and {\bf Y} using Jaccard
  Similarity.\\
\textcolor{red}{$ (7+4+11+2+9)/(1+8+10+3+6)= \frac{33}{28}$}

\item Compute the similarity between {\bf X} and {\bf Y} using Cosine
  Similarity.\\
\textcolor{red}{$\frac{(7*1)+(4*8)+(11*10)+(2*3)+(9*6)}{\sqrt{(7+4+11+2+9)}*\sqrt{(1+8+10+3+6)}}= 6.8755$}


\end{enumerate}


\newpage

\item (12 pts) Calculate the following values for functions commonly used
  by neural networks. Show all your work!

\begin{enumerate}

\item Consider a neural network with no hidden units, where X is
  the input vector and Y is a single output node. Let X = $<$.4 .3 .7$>$, the
  weight vector W = $<$.8 .2 .1$>$, and the bias weight b = .6.
  Compute the output value for Y using the sigmoid activation
  function. \\
  \textcolor{red}{
  $ Y = \frac{1}{1+e^{-DOT(W,X)+b}}$\\
  $DOT(W,X) = 0.45$\\
  $ Y = \frac{1}{1+e^{-.45+.6}}= 0.4625$\\}

\item Consider a neural network with no hidden units, where X is
  the input vector and Y is a single output node. Let X = $<$.2 .6 .1$>$, the
  weight vector W = $<$.5 .4 .8$>$, and the bias weight b = .3.
  Compute the output value for Y using the tanh activation
  function. \\
  \textcolor{red}{
  $ Y = \frac{e^z-e^{-z}}{e^z+e^{-z}}$\\
  $ z = DOT(W,X)+b = 0.42 +0.3 = 0.72$\\
  $ Y = \frac{e^{.72}-e^{-.72}}{e^{0.72}+e^{-0.72}}= 0.6169$\\}


\item Consider the following vector Z = $<$4.2 ~0.3 ~2.1 ~.75 ~.99$>$. 
Compute softmax(Z).  \\
\textcolor{red}{
$softmax(Z_1) = e^{4.2} / (e^{0.3} + e^{2.1} + e^{.75} + e^{.99}) = 4.65548$\\
$softmax(Z_2) = e^{0.3} / (e^{4.2} + e^{2.1} + e^{.75} + e^{.99}) = 0.01694$\\
$softmax(Z_3) = e^{2.1} / (e^{4.2} + e^{0.3} + e^{.75} + e^{.99}) = 0.11210$\\
$softmax(Z_4) = e^{.75} / (e^{4.2} + e^{0.3} + e^{2.1} + e^{.99}) = 0.02683$\\
$softmax(Z_5) = e^{.99} / (e^{4.2} + e^{0.3} + e^{2.1} + e^{.75}) = 0.03436$\\}


\end{enumerate}


\newpage

\item (13 pts) The questions below pertain to the LINKER
 event extraction system [Huang and Riloff,
  2012]. Consider the following story with five sentences:
\begin{quote}
S1: A small boy set the local market on fire. \\
S2: Police quickly found and arrested the young boy. \\
S3: The cops charged the suspect with arson. \\
S4: The young suspect will appear in court tomorrow.  \\
S5: A plea is expected by the suspect in court. \\
\end{quote}

\begin{description}
\item[(a)] Show the lexical bridge features that LINKER would create for
  the sentence pair S2+S3.  \\
  \textcolor{red}{$<$Police, the cops$>$ $<$found and arrested, charged$>$ $<$the young boy, suspect$>$  }


\item[(b)] Show the discourse focus features that the LINKER event
  extraction system would produce for the sentence pair S1+S2. \\
\textcolor{red}{$<$boy, object, subject$>$}

\item[(c)] Show the discourse focus features that the LINKER event
  extraction system would produce for the sentence pair  S4+S5.  \\
  \textcolor{red}{$<$suspect, PP(by), subject$>$}


\end{description}


Consider the following new story about an assassination that has six sentences:
\begin{quote}
S1: \underline{The mayor of Atlantis} was assassinated by gunmen. \\
S2: \underline{Walter Roswell} was shot yesterday.\\
S3: Atlantis has been plagued by violence and the deaths of
\underline{many innocent people} over the past decade. \\ 
S4: If more assassinations of \underline{politicians} occur this year, \underline{lawmakers} have
threatened to end public access to city hall.  \\
S5: The killing took place in front of \underline{several staff
  members}. \\
S6: \underline{Three armed men} entered city hall with rifles and shot
 Mayor Roswell.
\end{quote}

\begin{description}
\item[(d)] Assume that the underlined phrases in the story above were
labeled as {\it candidate role fillers} for the Victim role by the
role filler extractors. If the sentences S1, S2, S5, and S6 are
considered to be ``relevant'' sentences, list the phrases that would ultimately be
extracted by LINKER as Victims. \\


\textcolor{red}{$<$Victim$>$ was assassinated}
\textcolor{red}{$<$Victim$>$ was shot}
\textcolor{red}{in front of $<$Victim$>$ }
\end{description}


\newpage

\underline{\textbf{Question \#7 is for CS-6340 students ONLY!}}  \\

\item (10 pts) Suppose a word sense disambiguation (WSD) system is
  trying to distinguish between 3 senses of the word ``{\bf bug}'':
  $sense_A$ refers to an insect, $sense_B$ refers to a software error,
  and $sense_C$ refers to an obsession. Assume that a WSD system has
  disambiguated some instances of {\bf bug} in the documents below,
  labeling them as sense A, B, or C. (Note that these labels may not be correct!) 

\begin{quote}
\small
DOC \#1  \\
The computer {\bf bug/B} is probably in the print function. \\
The {\bf bug/B} is very subtle and may be hard to fix. \\
But we will fix the {\bf bug/?} by the end of the day! 
\end{quote}
\begin{quote}
\small
DOC \#2 \\
The {\bf bug/A} flew around the room and landed on a window sill. \\
John hit the {\bf bug/C} with a fly swatter.  \\
Miraculously, the {\bf bug/?} survived! 
\end{quote}
\begin{quote}
\small
DOC \#3 \\
A purple {\bf bug/A} was sitting on a flower. \\
The {\bf bug/B} was very colorful and pretty! \\
Mary suspected that the {\bf bug/A} was an exotic type of beetle. \\
Before she could check her book, the {\bf bug/?} flew away! 
\end{quote}
\begin{quote}
\small
DOC \#4 \\
George had been bitten by the writing {\bf bug/C} and hoped to be a novelist! \\
Last night when he was writing a book, a {\bf bug/A} crawled across the page. \\
He took the book outside and released the {\bf bug/A}.  \\
The {\bf bug/A} flew away and he resumed writing  for 12 hours! \\
George truly had a writing {\bf bug/?} and hoped he would become famous.
\end{quote}

\newpage
In each document, one instance of ``bug'' is unlabeled (marked
as {\bf bug/?}). Using Yarowsky's ``one sense per discourse''
heuristic, indicate whether the unlabeled instance of {\bf bug} in
each document should
be labeled as sense {\bf A}, {\bf B}, {\bf C}, or answer {\bf NONE} if it still cannot be labeled. 

\begin{description}
\item[(a)] What label should be assigned to {\bf bug/?} in DOC \#1?  \\
\textcolor{red}{B}

\item[(b)] What label should be assigned to {\bf bug/?} in DOC \#2?  \\
\textcolor{red}{NONE} 

\item[(c)] What label should be assigned to {\bf bug/?} in DOC \#3?  \\
\textcolor{red}{A} 

\item[(d)] What label should be assigned to {\bf bug/?} in DOC \#4?  \\
\textcolor{red}{C}


\item[(e)] Could Yarowsky's ``one sense per discourse'' heuristic be used to
{\it change} any of the original sense labels in DOC \#4? If YES, indicate which
instances of {\bf bug} should have their labels changed  and
\underline{briefly} explain why.   If not, just answer NO.
\textcolor{red}{"released the {\bf bug/A}" could be changed to /B as released a (software) bug could be a very common to have in the same sentance with the term (patch or build) release. diffrence would be verb noun. Yarowsky's one sense per }

\end{description}




\end{enumerate}


\end{document}
